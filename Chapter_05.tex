\chapter{Introduction}\label{CH_01}
(mostly rewritten from Bundy et al. 2015 - "The stripe82 massive galaxy project I"
Evolution of galaxies is a key theme in modern astrophysics. In our currently accepted cosmological model, dark matter halos (where subsequently galaxies and galaxy clusters have been formed) built up via hierarchical merging process. The physics of galaxy formation is much more complicated and involves gas interaction, stellar formation and various feedback processes (such as those due to supernovae and active galactic nuclei). An important observational constraint on galaxy formation theories is the global stellar mass density (GSMD), which quantifies the amount of baryonic matter that is locked into stars. 
\vspace{5mm} %5mm vertical space
Over several decades such estimates were made based either on a deep surveys over small area that suffer cosmic variance, or wide-field, much more shallow surveys that tend to pick up only brightest sources at medium redshifts (Malmquist bias). Existing Slaon Digital Sky Survey as well as new advancing surveys like DES, HSC, KiDS, DECaLS and VIKING together with new facilities such as LSST or Euclid should open a new era of deep panoramic multiband imaging with areas over thousands of square degrees.

These data shall be used in a numerous ways - from obtaining cosmological constraints to galaxy evolution and number density of objects in the early Universe, from measuring the growth of galaxies to observing the transition between different populations of galaxies at different epochs. Advance of the Big Data in astronomy requires new efficient  approach in combining information from different telescopes and different bands in a consistent, robust way.

The main goal of this thesis is to derive GSMD at low redshifts (up to z $\sim$ 0.8) using the largest sample possible to date. This is to be achieved by integrating the near-IR data from the Wide-field Infrared Survey Explorer (WISE) and the deep optical data from the Sloan Digital Sky Survey (SDSS) in the Stripe 82 region, totaling $\approx300 $ square degrees. We shall use unique approach to construct the most reliable spectral energy distribution (SED) spanning 0.35 to 4.6 $\mu$m. This large sample (more than XX million objects) will allow us to derive the GSMD in the low redshift universe to an unprecedented accuracy, and thus provide a critical reference for all galaxy formation models. As a by-product, we will also compile a large sample of optical-dropouts, which are objects detected in the near-IR by WISE but are missing from the SDSS in optical. The nature of these extremely red objects (ERO), \cite{Graham1996a}, \cite{Yan2004} is not well understood, and this unique sample will provide an important input catalog for future studies at the next generation facilities, such as the James Webb Space Telescope (JWST).

%Figure~\ref{fig:foo}
%\begin{figure}[!ht]
%\includegraphics[width=6in]{foo.ps}
%\caption{sample text}
%\label{fig:foo}
%\end{figure}

In this thesis, we use a $H_{0}=70 km s^{-1}Mpc^{-1}$, $\Omega_{\Lambda}=0.7$ and $\Omega_{m}=0.3$ cosmology, and assume that the distribution of the stellar mass follows Chabrier (2003) Initial Mass Function. Such IMF gives stellar masses consistent with Kroupa (2001) IMF to within 10\%.  