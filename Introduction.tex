\chapter{Introduction}\label{CH_Intro}

Evolution of galaxies is a one of the key topics in modern astrophysics. In our currently accepted cosmological model, dark matter halos (where galaxies and galaxy clusters have been formed and reside till now in a dark matter halo) are built up via hierarchical merging - a process in which larger structers are formed by continuous merging of a smaller-scale structures. The physics of galaxy formation is much more complicated and involves gas interaction, stellar formation and various feedback processes (such as those due to supernovae and active galactic nuclei). Over last two decades Lilly-Madau \citep{1996ApJ...460L...1L}, \citep{1996MNRAS.283.1388M} diagram was in the center of discussion - it shows that the star formation rate density (SFRD) in the Universe was increasing as we go to higher redshifts, peaks at $z\sim1$ and then decreases beyond that. Started with  only seven data points, three of which were low-limits, it was soon complemented by both high- and low-z data from a variety of data (e.g. \citep{2004ApJ...606L..25B}, \citep{2004ApJ...616L..79B}, \citep{2004MNRAS.355..374B}), most of which used Hubble Ultra-Deep field (UDF). While this diagram is still a widely-used tool to explain galaxy evolution, it is subject to uncertainties associated with robust estimation and calibration of star formation rates (SFR), effects of dust on SFR estimates, relatively poor sensitivity for most SFR indicators, and field-to-field variations caused by large scale structure (especially in a very small fields like UDF). For these reason it would be useful to have an independent way to understand star formation history of the Universe.

Global stellar mass density (GSMD), which quantifies the growth in time of the amount of baryonic matter that is locked into stars, gives us such opportunity - it equals to the integral over the cosmic SFR, i.e. contains the same information about evolution of the galaxies (assuming correct choice of IMF), but suffers from different systematic uncertainties and is therefore an invaluable probe of the broad evolution of the stellar content of the Universe. Recent works by \citep{2013ApJ...767...50M}, \citep{2012A&A...545A..23B}, \citep{2012A&A...545A..23B}, \citep{2013ApJ...777...18M} (more results are presented in a review paper by \citep{Madau2014}) significantly contributed to our knowledge of the shape of GSMD up to redshifts $z\approx8$. But the picture is not clear - the spread in results can reach up to 0.5 dex even in the local Universe ($z<0.1$). \\

The main goal of this thesis is to derive GSMD at low redshifts (up to z $\sim0.8$) using the largest sample possible to date. This is to be achieved by integrating the near-IR data from the Wide-field Infrared Survey Explorer (WISE) \citep{Wright2010} and the deep optical data from the Sloan Digital Sky Survey in the Stripe 82 region, totaling $\approx300 $ square degrees. We shall use unique approach to construct the most reliable spectral energy distribution (SED) spanning 0.35 to 4.6 $\mu m$. This large sample (more than 9 million galaxies) will allow us to derive the GSMD in the low redshift universe to an unprecedented accuracy. In the era of advancing new generation IR facilities like James Webb Space Telescope (JWST) and WFIRST, our estimates provide a critical reference for all galaxy formation models.\\

As a by-product, we also compile a large sample of optical-dropouts that we call "WISE Optical Dropouts" or "WoDrops", which are objects detected in the near-IR by WISE but are missing from the SDSS in optical. The nature of these extremely red objects (ERO), \citep{Graham1996a}, \citep{Yan2004} is not well understood, and this unique sample will provide an important input catalog for future studies at the next generation telescopes, including JWST.\\

The structure of this thesis is as follows:
in Chapter 2 we present the data sets that we use in our study and also some preliminary operations with images that ought to be done in order to construct sources catalogs. Construction of such catalogs and problems associated with it are discussed in Chapter 3. SED fitting that allows us to derive physical properties of the galaxies is presented in Chapter 4. We use Chapter 5 to discuss our final results - evolution of stellar mass density over last 6 Gyr and compare our results to other groups. Chapter 6 concludes our work, we put our results in the context of the study of the galaxy evolution, describe future work, including a study on "WoDrops".\\

Throughout the thesis we use a $H_{0}=70 km s^{-1}Mpc^{-1}$, $\Omega_{\Lambda}=0.7$ and $\Omega_{m}=0.3$ cosmology, and assume that the distribution of the stellar mass follows \citep{2003PASP..115..763C} initial mass function (IMF). Such IMF gives stellar masses consistent with \citep{2001MNRAS.322..231K} IMF to within 10\% and are different from \citep{1955ApJ...121..161S} IMF by a constant factor 1.64.

%Figure~\ref{fig:foo}
%\begin{figure}[!ht]
%\includegraphics[width=6in]{foo.ps}
%\caption{sample text}
%\label{fig:foo}
%\end{figure}

%\fnurl{name}{http://site}



