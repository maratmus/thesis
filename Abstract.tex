\newpage
\phantomsection\addcontentsline{toc}{chapter}{ABSTRACT}

\centerline{\bf \large ABSTRACT}
\vspace{10mm} % Edit everything below with your acknowledging text.
In this dissertation, I aim to study the evolution of galaxies over the last 6 Gyr by measuring the growth of the global stellar mass density (GSMD) since $z=0.8$. My work combines the datasets from two very large surveys, namely, the optical data from the Sloan Digital Sky Survey (SDSS) Stripe 82 and the infrared data from the Wide-field Infrared Survey Explorer (WISE), and constructs a unique catalog of galaxies that have their spectral energy distributions (SEDs) measured consistently from 0.3 to 5~$\mu$m in seven bands. This catalog, the largest of its kind, contains 9 million galaxies in $\sim300~deg^2$, will have a wide range of applications beyond the scope of this thesis.\\

Extending galaxy SED measurements to restframe near-IR has two significant advantages: (1) dust extinction can be better handled, and (2) emissions from low-mass stars, which are the major contributors to a galaxy's stellar mass, can be better measured. WISE was the only mission to date that provided all-sky IR data that are deep enough  for galaxy evolution studies out to $z\approx 1$ (sampling restframe K-band). The only wide-field optical survey data that could match WISE depths are those from the SDSS Stripe 82 over $\sim 300$ deg$^2$. The synergy of the two is therefore natural. The implementation, however, is of tremendous difficulty. This is mainly because of the vastly different spatial resolutions between SDSS and WISE. To overcome this problem, we take an approach that is often referred to as "morphological template fitting", i.e., using the high-resolution image to define the morphological template of the galaxy in question, and de-convolving its light profile in the low-resolution image accordingly. In this way, we obtain the SED measurements over the entire 0.3-5$\mu$m range in the most self-consistent manner. Using this SED catalog as the basis, we derive photometric redshifts and stellar masses for all the 9 million galaxies that span $z=0$-0.8. This provides us an unprecedented statistics when deriving galaxy stellar mass functions  (MFs) and GSMD over multiple redshift bins. Some preliminary results are discussed. \\

As a by-product of our morphological template fitting process, an interesting population of objects called "WISE Optical Dropouts" ("WoDrops" for short) are discovered. These objects are significant detections in WISE data but are invisible in all the SDSS Stripe 82 data. Their nature remains a mystery up to this point. Among all possibilities, the only viable interpretation is that they are very high-mass galaxies with very high dust extinctions. To reveal their nature, future observations at larger facilities will be necessary.